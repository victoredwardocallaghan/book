\section{Dual Category} % (fold)
\label{sec:dualcategory}
$\emph{Duality}$ is a correpsondence between properties of a category $\mathcal{K}$ and the notion
of $\emph{dual properties}$ of the ``$\emph{op}$posite category'' $\mathcal{K}^{op}$. Suppose some proposition
regarding a category $\mathcal{K}$, by interchanging the domain and codomain of each morphism as well as
the order of composition, a corresponding dual proposition is obtained of $\mathcal{K}^{op}$.

Duality, as such, is the assertion that truth is invariant under this operation on propositions.
\begin{lem}
 $(\mathcal{K}^{op})^{op} = \mathcal{K}$.
\end{lem}

\begin{exmp}
 Suppose a monomorphism in a category $\mathcal{K}$ then an epimorphism is the categorial dual
 in the dual category $\mathcal{K}^{op}$. Prove this by diagram.
\end{exmp}

\begin{rem}
 We make the notion of diagram and hence the idea of commutative diagrams rigous later.
\end{rem}

\subsection{Coproduct} % (fold)
\label{subsec:coproduct}
Let $A,B \in \text{Ob}(\mathcal{K})$ for some category $\mathcal{K}$.
A $\emph{co}$product (sum) of $A$ and $B$ is an object $C$, together with morphisms from
$\alpha: A \to C$ and $\beta: B \to C$, such that the following property holds:
if $C^{'}$ is any object, and any morphisms $\alpha^{'}: A \to C^{'}$ and $\beta^{'}: B \to C^{'}$,
there is a unique morphism $\gamma: C \to C^{'}$ such that the following diagram commutes:

\begin{tikzcd}
A \ar{ddr}[swap]{\alpha^{'}} \drar{\alpha}
 &&
 B \ar{ddl}{\beta^{'}} \dlar[swap]{\beta} & \\
 & C \dar{\gamma} & \\
 & C^{'}
\end{tikzcd}

Once again, the morphisms $\alpha$ and $\beta$ are called the $\emph{canonical projections}$.
We denote the coproduct, or `sum', of $A$ and $B$ as: $C = A \coprod B$ or sometimes written $C = A \bigoplus B$.
Notice that all we have done is reverse all the morphisms (arrows) in a product to obtain the coproduct.
This is, ofcourse, the usual method of finding a categorial dual.