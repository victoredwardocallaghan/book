%
% Copyright © 2013 Edward O'Callaghan. All Rights Reserved.
%

\section{Groupoids}
\label{sec:groupoids}

The extension from \emph{group} to \emph{groupoid} is motivated with the
desire to describe reversible mappings which may traverse a number of states.
Intuitively, in \emph{group theory} we study mappings of the form
$\tau : \mathcal{G} \times \mathcal{G} \to \mathcal{G}$
that are total functions. Whereas in \emph{groupoid theory} one considers a set
of compositable and invertible mappings of the form
$\phi \circ \psi : \mathcal{G} \times \mathcal{G} \to \mathcal{G}$
where $\phi$ and $\psi$ need only be partial functions on $\mathcal{G}$.

\begin{defn}[Groupoid]
 \label{defn:groupoid}

 A \define{groupoid} is a small category in which every morphism is an isomorphism.

 \indexdef{groupoid}
 \indexsee{groupoid}{groupoid}
\end{defn}

%\begin{exmp}
Thus, as a result one may view groups as a groupoid with one object.
More precisely, a groupoid $X$ with one object $x$ is uniquely determined
by its automorphism group $\mathcal{G}=\text{Aut}(x)$. Consequently, any group
may be regarded as a groupoid with one object.

Notice that can naturally view morphisms between \emph{groupoids} are
\emph{functors} between categories.

