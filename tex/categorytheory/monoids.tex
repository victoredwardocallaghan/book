\section{Monoids} % (fold)
\label{sec:monoids}
A $\emph{monoid}$ is an abstract mathematical structure normally associated with that of a semigroup with identity.
A semigroup with identity, or just monoid, is some set $M$ together with a law of associative composition,
$\forall x,y,z \in M : (x \circ y) \circ z = x \circ (y \circ z)$, that is both closed, $M \circ M \to M$,
and has idenity, $\exists \Id{M} \in M : \Id{M} \circ x = x = x \circ \Id{M} \, \forall x \in M$. That is,

\begin{defn}
 A $\emph{monoid}$ is a set $X$ together with a law of composition $\circ$
 which is closed associative, and has identity $\Id{X} \in X$.
\end{defn}

More compactly in categorial terms,
\begin{defn}[Monoid]
 A category with exactly one object is called a $\emph{monoid}$.
\end{defn}

\begin{exmp}
 Let $M = \Z$ and the law of composition $+$ as arithmetic addition with $0$ as identity.
\end{exmp}

\begin{exmp}
 A list is an example of a monoid with the empty list $()$ as identity and the law of composition $++$ as appending to the list.
\end{exmp}


% \begin{exmp}
%  The following commutative diagram illustrates the monoidal structure of a group:
%  \monoid{
%   (A\otimes A)\otimes A\otimes A &
%   A\otimes (A\otimes A) &
%   A\otimes A &
%   A\otimes A &
%   A &
%   \mu\otimes \Id{} &
%   \Id{} \otimes\mu &
%   \mu &
%   \mu\otimes \Id{}
%  }
% \end{exmp}