%
% Copyright © 2013 Edward O'Callaghan. All Rights Reserved.
%

\section{Set theory} % (fold)
\label{sec:settheory}

It has been proposed that the framework of \emph{set theory} is sufficient to
support the foundations of mathematics. We shall revise this later by lifting
mathematics to a new modern and very actively researched framework. However,
for now, we digress and consider set theory as it is of course no less useful.

Let us consider some of set theories histrionics for it contains rich lessons
in logical failles not to be forgotten. Set theory in its early development had
very loss definitions of what exactly is a \emph{set}. The inadequacy of the
loss definition of a set as simply a `collection of objects` becomes blindingly
apparent with the logical inconsistencies that arose such as the famous
"Russell's paradox".
% cite:
% George Cantor
% Burali-Forti
% Burtrand Russell

In light of these logical inconsistencies it became apparent that such notions
of what constitutes a \emph{set} must be defined more meticulously. However,
since the very notion of a set, due in part to its simplicity, is outside scope
it is by is very nature undefinable in a rigours manner. Thus sets are
axiomatic and hence outside the scope of `logical provability`, latter
formalised as the \emph{incompleteness theorem}. Serveral axiomatic systems
were however devised to deal with the logical inconsisances, the most popular
of which is the ZFN system.
% cite: systems -
% ZFN
% GHB
% RW
..

\begin{defn}
 A object $x$ contained in a set $X$ is said to be an \emph{element} of the
 set, denoted $x \in X$. The logical negation $\neg (x \in X)$ is written
 $x \not \in X$ and read as $x$ is \emph{not} an element of the set $X$.
\end{defn}

..

\begin{defn}
 Equality of two sets $X$ and $Y$ is defined by; $X=Y$ if and only if
 $X \subseteq Y$ and $Y \subseteq X$.
\end{defn}

Set \emph{intersections} can be understood as a generalisation of predicate
logic "and" conjunction written $\wedge$ between elements to intersections
between sets of elements written $\cap$. More formally;

\begin{defn}
 The \emph{intersections} of two sets $X$ and $Y$ is defined as the set,
 $X \cap Y = \{ x \in X \wedge y \in Y \}$.
\end{defn}

\begin{lem}
 Set intersection \emph{idempotent} property $X \cap X = X$.
\end{lem}

Set \emph{unions} can then be understood as the logical negation of
intersections. Hence the logical "or" disjunction written $\vee$ between
elements to unions between sets of elements written $\cup$. More formally;

\begin{defn}
 The \emph{unions} of two sets $X$ and $Y$ is defined as the set,
 $X \cup Y = \{ x \in X \vee y \in Y \}$.
\end{defn}

\begin{lem}
 Set union \emph{idempotent} property $X \cup X = X$.
\end{lem}

In symbols, recall that $x \wedge y$ for booleans variables $x,y$ then leads us
to $X \cap Y$ for sets $X,Y$. So, since $\neg (x \wedge y)= \neg x \vee \neg y$
we have that $\neg (x \in X \cap y \in Y) = \neg (x \in X) \cup \neg (y \in Y)$.
We then adopt a more general notation for set logical negations called the
\emph{complement} written $X^c$ defined in the following way;
If $\neg (x \in X)$ then $x \not \in X$ and so for every $x \not \in X$ we form
the complement set written $X^c$ for which $x \in X^c$ whenever $x \not \in X$.

Hence we arrive at DeMorgan's laws:
\begin{itemize}
 \item $(X \cap Y)^c = X^c \cup Y^c$ and
 \item $(X \cup Y)^c = X^c \cap Y^c$.
\end{itemize}

